\section{实验中的变量}
变量的选择能否回答想解决的问题?
符合:1研究的理论假说/实验设计和统计的要求

\textbf{被试固有特性}:被试在实验前就有的,不太可能改变和操纵,通常分类
严格地来说,追踪的研究能得到严格的因果关系吗?
最严格的因果关系是操纵和改变自变量,引起心理或行为上的变化
有些情况下没有办法做这个
但是又没有办法做,已经是最好的了
通常在实验中还会加一个可以操控的,不会让自变量只有被试固有特性,

\textbf{被试的暂时特性}:心理学严重最丰富,最具特色的自变量.操纵外部刺激引起心理变化,不是完全外部,引起的心理变化正好是想研究的问题,但是比较难以操纵,需要找到非常好的操作定义.

可操作、可改变的
不可操作、不可改变的

自变量的数量:与问题、统计方法有关
水平的数量:动机的那个例子

主要意义次要次要意义,定SOA,43
主要意义一直维持激活,次要意义有一个上升下降的曲线


因变量:可以看到的
必须随着自变量的变化而变化
可以转化为数据

1)对自变量的变化最敏感:字典中
2)实验中选取的观测量应该可靠,得出稳定的结果
3)大体上正态分布,人的行为、生理大体上都符合正态分布,但是如果数据就不可能服从正态分布就不要用方差分析
4)可行性上考虑

Keenan(2001)自我面孔实别

无关变量

操作定义:变量必须用测量或操作它们的步骤来定义
用事物可观察、可测量的特征把研究变量具体化,它清楚地讲明与栽些现象独特联系的可观测的准则

意义:
1、对研究结果评价:同意研究结果必须先接受操作定义
2、有利于研究间交流:在前人基础上重复或修改实验
3、保障可重复性

获得方式:
1、使用现有信息:性别、教育程度、智力
2、操作创建一种形态:让被试进入实验后让其脑子发生变化,最重要
3、通过前测、评定、记录获得信息,如语义相关

Rosler(2001)

Loftus(1974)看相同的视频, 只是问题的词不同
The possible speed when the two cars