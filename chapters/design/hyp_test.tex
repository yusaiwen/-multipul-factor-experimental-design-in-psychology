\section{假设检验的基本思想}

要先说说统计的三大分布,这部分内容在概率中有详细论述,因此我也不打算说的很详细.

1. $\Gamma$函数

$$
\Gamma \left( x \right) =\int_0^{\infty}{e^{-t}t^{x-1}dt\left( x>0 \right)}
$$

性质:

\begin{align*}
    \text{(1) } & \Gamma \left( x+1 \right) =x\Gamma \left( x \right)  \qquad
    \Gamma \left( 1 \right) =1\Rightarrow \Gamma \left( n+1 \right) =n! \\
    \text{(2) } & \Gamma \left( \frac{1}{2} \right) =\sqrt{\pi}
\end{align*}


2.$\beta$函数
$$
\beta \left( x,y \right) =\int_0^1{t^{x-1}\left( 1-t \right) ^{y-1}dt\left( x>0,y>0 \right)}
$$
则有	 
$$
\beta \left( x,y \right) =\frac{\Gamma \left( x \right) \Gamma \left( y \right)}{\Gamma \left( x+y \right)}
$$

1.$\chi ^2$分布(Chi-Square Distribution)
$$
\text{设}X_1, X_2,..., X_n\,\,iid,\sim N\left( 0,1 \right) , \text{令}\chi _{n}^{2}=\sum_{i=1}^n{X_{i}^{2}}
$$

称$\chi _n^2$是自由度为$n$的$chi$方分布,其密度函数为 :
$$
f_n\left( x \right) =\frac{1}{\Gamma \left( \frac{1}{n} \right) 2^{\frac{n}{2}}}\cdot e^{-\frac{x}{2}}\cdot x^{\frac{n-2}{n}}\left( x>0 \right) 
$$

$chi$方分布有一个重要的性质:设$X_1\,\,,X_2$独立,$X_1\sim \chi _{m}^{2}, X\sim \chi _{n}^{2}$,则$X_1+X_2\sim \chi _{m+n}^{2}$ 

2.$t$分布
设$X\sim N\left( 0,1 \right) , Y\sim \chi _{n}^{2}$ ,且$X,Y$独立,于是

$$
t_n=\frac{X}{\sqrt{Y/n}}\left( \frac{N\left( 0,1 \right)}{\sqrt{\chi _{n}^{2}/n}} \right) 
$$

称t服从自由度为$n$的$t$分布,$t$分布的自由度与其分母的¥方自由度相等,其密度函数为: 
$$
f_n\left( t \right) =\frac{\Gamma \left( \frac{n+1}{2} \right)}{\sqrt{n}\pi \Gamma \left( \frac{n}{2} \right)}\left( 1+\frac{t^2}{n} \right) ^{-\frac{n+1}{2}}
$$

$t$分布图像如所示,可以看到$t$分布与正态分布在一起时,$t$分布的形态好像“尾巴非常重,头变轻”,故$t$分布有也被叫做heavy tail(重尾).由图像亦可知,$t$分布随着自由度的变大,“头上的鼓包”渐渐缩小,本来“繁重的尾巴”渐渐变轻,越来越朝着标准正态分布的图像靠近.

3.$F$分布

设$X\sim \chi _{m}^{2},Y\sim \chi _{n}^{2}$,且$X$与$Y$独立,令 ,称$F$为服从自由度$m,n$的$F$分布.
当$m=1$时
$$
F_{1,n}=\frac{\chi _{1}^{2}/1}{\chi _{n}^{2}/n}=\left( \frac{N\left( 0,1 \right)}{\sqrt{\chi _{n}^{2}/n}} \right) ^2=t_{n}^{2}
$$
故
$$
t_{n}^{2}\sim F_{1,n}
$$

F分布的基本假设

因变量观测值大体服从正态分布,如果数据不可能符合正态分布,就不能用F检验
 
变异的同质性:由随机分配保证,是一个最基本的假设,因为我们根本就不做前测,只做后测
	加实验处理只是变了组间变异,实验后的组间变异也反应了处理前的组内变异
 
 独立性:每一个观测值独立于其它观测值 


实验设计模型及其假设:变异的构成
一次观测就只有一个观察值,但是怎么知道这个观察值有哪些组成的?这个模型就是指导这件事,
如:
$$
Y_{ij}=\mu +\alpha_j + \varepsilon_{i(j)}
$$

$$
SS_{total}=SS_{within\_group}+SS_{between\_group}
$$



误差是一个平均数为0的随机变量,不是一个定向的变化


实验设计的基本过程:
1提出问题和假说的形成
问题特征:探索两个或多个变量间的关系
问题的可检验性:研究该问题的方法和手段已经解决或初步解决
Science 290 ,1582-1585(2000) 有意思的问题,在以前是没法做的,婴儿没法报告
问题所包含的变量是可以检验并可以操作化的(操作定义)
怎么操作世界知识违反操作化?非常精巧化