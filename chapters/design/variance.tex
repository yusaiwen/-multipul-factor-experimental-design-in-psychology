\section{实验中变异的控制}
所谓变异(variance),就是对数据离散情况的衡量.和方差(variance)是一个词,不过实验中的变异含有三种变异

\textbf{1.使系统变异最大}
	\begin{itemize}
		\item 选取适当的自变量水平,使自变量水平的改变所引起的变异能在因变量中反映出来
		\item 选择对自变量的变化敏感的因变量
	\end{itemize}

这个实验中,一个是10,000条,一个是20,000条
因变量:容易形成正态分布
愿意买哪本?1010的数据,变化不敏感,变异小
%-----------------------------------------------------------------
\textbf{2.控制无关变异}
所有可能做自变量的因素均可能成为因素均可能成为无关变异的来源,无关变量是心理学、教育学、社会学研究中最棘手的问题之一.

上面大:例如研究两种教学条件对成绩的影响,实验选择两个班,但是可能两个班的学生能力不同,使得处理效应变异虚大
下面大:主要是被试带来的个体差异

这个实验中随机分配
指导语的描述一模一样,由不同文字引起的带走了

有五种控制无法变异的基本方法:

\begin{itemize}
	\item 随机化:随机分配被试
	\item 消除:i不是一个好方法,比如性别会影响,就只研究男性
	\item 匹配:基本上都是一样的
	\item 附加自变量
	\item 统计控制
\end{itemize}
%-----------------------------------------------------------------
\textbf{3.使误差变异最小}
误差指的是在实验中没有控制的变异:
主要来自于两个:处理单元内被试间的个体差异,我们知道这个是行为学实验中最主要的差异,
测量误差
尽量在实验设计中控制,实在控制不了的用统计控制 

实验设计的评价:内部效度(稳定性、控制无法变异)、外部效度