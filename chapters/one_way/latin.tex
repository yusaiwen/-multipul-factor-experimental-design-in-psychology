\section{单因素拉丁方实验设计}

\subsection{基本特点}

拉丁方设计最大特色是同时计算了残差和单元内误差,扩展了区组实验设计.对于完全随机方差分析中的$F$检验而言:
\[ F=\frac{MSA}{MS_{\text{组内}}} \qquad MS_{\text{组内}}=\text{无关变异}+\text{随机误差} \]

由于完全随机设计中误差项$MSE$中混入了个体差异和一些其它的无关变异,所以随机区组设计找到一个影响较大的无关变异,计算其大小并从$MSE$中分离,这样新的$MSE$就等于$MSE_{\text{组内}}-MS_{\text{区组}}$,误差项减小,可以提高实验的精度.
不过这样还不够,首先,对于影响因变量的无关变量不止有一个,也许除了生字密度学、生的智力外,还有别的无关变量对阅读理解有较大影响.故我们希望可以再分离出一个无关变量的效应;
其次,随机区组的误差是否是一个合理的误差也不清楚
所以我还希望在每一个区组内加入多个被试,以检验残差的是否显著.

总的来说,拉丁方实验设计同时分离两个无关变量的效应,同时通过分配多个被试给同一自变量水平和无关变量水平的结合,使得可以计算出单元内误差来检验残差是否显著.不过该设计对两个无关变量有较大要求.拉丁方实验设计适合下列条件:

\begin{description}
\item[1.自变量和无关变量的水平限制] 自变量的水平和两个无关变量的水平相等,即自变量水平是$p$ $(p\geq 2)$,两个无关变量水平都是$p$ $(p\geq 2)$;
\item[2.自变量与无关变量无交互作用] 如果这个假设不能满足峄实验中的一个或多个效应的检验可能有偏差
\item[3.拉丁方限制]随机分配处理立平给$p^2$个方格单元,每个处理水平仅在每行、每列中出现一次
\end{description}

%-----------begin连续的拉丁方快
\begin{margintable}
  \caption{$2\times 2$拉丁方快}
  \labtab{lantin_square_2_2}
    \begin{tabular}{cc}
    A     & B \\
    B     & A \\
    \end{tabular}
\end{margintable}

\begin{margintable}
  \caption{$3\times 3$拉丁方快}
  \labtab{lantin_square_3_3}
    \begin{tabular}{ccc}
    A  &  B  &  C\\
    B  &  C  &  A\\
    C  &  A  &  B\\
    \end{tabular}
\end{margintable}

\begin{margintable}
  \caption{$4\times 4$拉丁方快}
  \labtab{lantin_square_4_4}
    \begin{tabular}{cccc}
    A     & B     & C     & D \\
    B     & C     & D     & A \\
    C     & D     & A     & B \\
    D     & A     & B     & C \\
    \end{tabular}
\end{margintable}

\begin{margintable}
  \caption{$5\times 5$拉丁方快}
  \labtab{lantin_square_5_5}
    \begin{tabular}{ccccc}
    A     & B     & C     & D     & E \\
    B     & C     & D     & E     & A \\
    C     & D     & E     & A     & B \\
    D     & E     & A     & B     & C \\
    E     & A     & B     & C     & D \\
    \end{tabular}
\end{margintable}
%---------------------end连续的拉丁方快

我列出了五个拉丁方格:\vreftab{lantin_square_2_2},\vreftab{lantin_square_3_3},\vreftab{lantin_square_4_4},\vreftab{lantin_square_5_5},从列子中可以看到拉丁方格每一行每一列每一个字母只出现一次,下面以$4\times 4$拉丁方表格介绍一下拉丁方格随机化的方法:

\begin{table}[h]
  \caption{拉丁方格标准化方块的随机化}
  \labtab{lantin_square_norm}
    \begin{tabular}{ccccccccccc}
          & \multicolumn{4}{c}{标准块}       &       &       & \multicolumn{4}{c}{随机化行} \\
          & 1     & 2     & 3     & 4     &       &       & 1     & 2     & 3     & 4 \\
    1     & \cellcolor[rgb]{ .851,  .882,  .949}A & \cellcolor[rgb]{ .851,  .882,  .949}B & \cellcolor[rgb]{ .851,  .882,  .949}C & \cellcolor[rgb]{ .851,  .882,  .949}D &       & 3     & \cellcolor[rgb]{ .886,  .937,  .855}C & \cellcolor[rgb]{ .886,  .937,  .855}D & \cellcolor[rgb]{ .886,  .937,  .855}A & \cellcolor[rgb]{ .886,  .937,  .855}B \\
    2     & \cellcolor[rgb]{ 1,  .949,  .8}B & \cellcolor[rgb]{ 1,  .949,  .8}C & \cellcolor[rgb]{ 1,  .949,  .8}D & \cellcolor[rgb]{ 1,  .949,  .8}A &       & 1     & \cellcolor[rgb]{ .851,  .882,  .949}A & \cellcolor[rgb]{ .851,  .882,  .949}B & \cellcolor[rgb]{ .851,  .882,  .949}C & \cellcolor[rgb]{ .851,  .882,  .949}D \\
    3     & \cellcolor[rgb]{ .886,  .937,  .855}C & \cellcolor[rgb]{ .886,  .937,  .855}D & \cellcolor[rgb]{ .886,  .937,  .855}A & \cellcolor[rgb]{ .886,  .937,  .855}B &       & 2     & \cellcolor[rgb]{ 1,  .949,  .8}B & \cellcolor[rgb]{ 1,  .949,  .8}C & \cellcolor[rgb]{ 1,  .949,  .8}D & \cellcolor[rgb]{ 1,  .949,  .8}A \\
    4     & \cellcolor[rgb]{ .929,  .929,  .929}D & \cellcolor[rgb]{ .929,  .929,  .929}A & \cellcolor[rgb]{ .929,  .929,  .929}B & \cellcolor[rgb]{ .929,  .929,  .929}C &       & 4     & \cellcolor[rgb]{ .929,  .929,  .929}D & \cellcolor[rgb]{ .929,  .929,  .929}A & \cellcolor[rgb]{ .929,  .929,  .929}B & \cellcolor[rgb]{ .929,  .929,  .929}C \\
          &       &       &       &       &       &       &       &       &       &  \\
          & \multicolumn{4}{c}{随机化行}      &       &       & \multicolumn{4}{c}{随机化列} \\
          & 1     & 2     & 3     & 4     &       &       & 4     & 3     & 1     & 2 \\
    3     & \cellcolor[rgb]{ .851,  .882,  .949}C & \cellcolor[rgb]{ .988,  .894,  .839}D & \cellcolor[rgb]{ .929,  .929,  .929}A & \cellcolor[rgb]{ .886,  .937,  .855}B &       & 3     & \cellcolor[rgb]{ .886,  .937,  .855}B & \cellcolor[rgb]{ .929,  .929,  .929}A & \cellcolor[rgb]{ .851,  .882,  .949}C & \cellcolor[rgb]{ .988,  .894,  .839}D \\
    1     & \cellcolor[rgb]{ .851,  .882,  .949}A & \cellcolor[rgb]{ .988,  .894,  .839}B & \cellcolor[rgb]{ .929,  .929,  .929}C & \cellcolor[rgb]{ .886,  .937,  .855}D &       & 1     & \cellcolor[rgb]{ .886,  .937,  .855}D & \cellcolor[rgb]{ .929,  .929,  .929}C & \cellcolor[rgb]{ .851,  .882,  .949}A & \cellcolor[rgb]{ .988,  .894,  .839}B \\
    2     & \cellcolor[rgb]{ .851,  .882,  .949}B & \cellcolor[rgb]{ .988,  .894,  .839}C & \cellcolor[rgb]{ .929,  .929,  .929}D & \cellcolor[rgb]{ .886,  .937,  .855}A &       & 2     & \cellcolor[rgb]{ .886,  .937,  .855}A & \cellcolor[rgb]{ .929,  .929,  .929}D & \cellcolor[rgb]{ .851,  .882,  .949}B & \cellcolor[rgb]{ .988,  .894,  .839}C \\
    4     & \cellcolor[rgb]{ .851,  .882,  .949}D & \cellcolor[rgb]{ .988,  .894,  .839}A & \cellcolor[rgb]{ .929,  .929,  .929}B & \cellcolor[rgb]{ .886,  .937,  .855}C &       & 4     & \cellcolor[rgb]{ .886,  .937,  .855}C & \cellcolor[rgb]{ .929,  .929,  .929}B & \cellcolor[rgb]{ .851,  .882,  .949}D & \cellcolor[rgb]{ .988,  .894,  .839}A \\
    \end{tabular}%
\end{table}

单因素拉丁方实验设计被试分配到处理上的例子如\vreftab{latin_subject_treatment}.从表中可以看出,实验中的自变量$A$有4个水平,无关变量$B$和无关变量$C$也各有4个水平,形成$4\times 4$的拉丁方格,32个被试参加了实验,每个方格内有2个被试,每个被试只接受一种独特的实验条件处理.

\begin{margintable}
  \centering
  \caption{单因素拉丁方实验设计中被试的分配}
    \begin{tabular}{ccccc}
          & $c_1$ & $c_2$ & $c_3$ & $c_4$ \\
    \multirow{3}[0]{*}{$b_1$} & \cellcolor[rgb]{ .851,  .882,  .949}$a_1$ & \cellcolor[rgb]{ .851,  .882,  .949}$a_2$ & \cellcolor[rgb]{ .851,  .882,  .949}$a_3$ & \cellcolor[rgb]{ .851,  .882,  .949}$a_4$ \\
          & \cellcolor[rgb]{ .886,  .937,  .855}$S_1$ & \cellcolor[rgb]{ .886,  .937,  .855}$S_9$ & \cellcolor[rgb]{ .886,  .937,  .855}$S_{17}$ & \cellcolor[rgb]{ .886,  .937,  .855}$S_{25}$ \\
          & \cellcolor[rgb]{ .886,  .937,  .855}$S_2$ & \cellcolor[rgb]{ .886,  .937,  .855}$S_{10}$ & \cellcolor[rgb]{ .886,  .937,  .855}$S_{18}$ & \cellcolor[rgb]{ .886,  .937,  .855}$S_{26}$ \\
    \multirow{3}[0]{*}{$b_2$} & \cellcolor[rgb]{ .851,  .882,  .949}$a_2$ & \cellcolor[rgb]{ .851,  .882,  .949}$a_3$ & \cellcolor[rgb]{ .851,  .882,  .949}$a_4$ & \cellcolor[rgb]{ .851,  .882,  .949}$a_1$ \\
          & \cellcolor[rgb]{ .886,  .937,  .855}$S_3$ & \cellcolor[rgb]{ .886,  .937,  .855}$S_{11}$ & \cellcolor[rgb]{ .886,  .937,  .855}$S_{19}$ & \cellcolor[rgb]{ .886,  .937,  .855}$S_{27}$ \\
          & \cellcolor[rgb]{ .886,  .937,  .855}$S_4$ & \cellcolor[rgb]{ .886,  .937,  .855}$S_{12}$ & \cellcolor[rgb]{ .886,  .937,  .855}$S_{20}$ & \cellcolor[rgb]{ .886,  .937,  .855}$S_{28}$ \\
    \multirow{3}[0]{*}{$b_3$} & \cellcolor[rgb]{ .851,  .882,  .949}$a_3$ & \cellcolor[rgb]{ .851,  .882,  .949}$a_4$ & \cellcolor[rgb]{ .851,  .882,  .949}$a_1$ & \cellcolor[rgb]{ .851,  .882,  .949}$a_2$ \\
          & \cellcolor[rgb]{ .886,  .937,  .855}$S_5$ & \cellcolor[rgb]{ .886,  .937,  .855}$S_{13}$ & \cellcolor[rgb]{ .886,  .937,  .855}$S_{21}$ & \cellcolor[rgb]{ .886,  .937,  .855}$S_{29}$ \\
          & \cellcolor[rgb]{ .886,  .937,  .855}$S_6$ & \cellcolor[rgb]{ .886,  .937,  .855}$S_{14}$ & \cellcolor[rgb]{ .886,  .937,  .855}$S_{22}$ & \cellcolor[rgb]{ .886,  .937,  .855}$S_{30}$ \\
    \multirow{3}[0]{*}{$b_4$} & \cellcolor[rgb]{ .851,  .882,  .949}$a_4$ & \cellcolor[rgb]{ .851,  .882,  .949}$a_1$ & \cellcolor[rgb]{ .851,  .882,  .949}$a_2$ & \cellcolor[rgb]{ .851,  .882,  .949}$a_3$ \\
          & \cellcolor[rgb]{ .886,  .937,  .855}$S_7$ & \cellcolor[rgb]{ .886,  .937,  .855}$S_{15}$ & \cellcolor[rgb]{ .886,  .937,  .855}$S_{23}$ & \cellcolor[rgb]{ .886,  .937,  .855}$S_{31}$ \\
          & \cellcolor[rgb]{ .886,  .937,  .855}$S_8$ & \cellcolor[rgb]{ .886,  .937,  .855}$S_{16}$ & \cellcolor[rgb]{ .886,  .937,  .855}$S_{24}$ & \cellcolor[rgb]{ .886,  .937,  .855}$S_{32}$ \\
    \end{tabular}%
  \labtab{latin_subject_treatment}
\end{margintable}

\subsubsection{单因素拉丁方设计模型}
\begin{definition}[单因素拉丁方设计模型]
\labdef{one_way_latin_model}
\begin{align*}
    Y_{ijkl} = \mu + \alpha _j + \beta _k & + \gamma _l + \varepsilon _{pooled}\\
                                          &(j=1,2,\cdots,p;i=1,2,\cdots,n)\\
                                          &(k=1,2,\cdots,p;l=1,2,\cdots,p)
\end{align*}

其中

{
    \renewcommand\arraystretch{1.25}
    \begin{tabular}{ccl}
        $Y_{ijkl}$     & - &    两无关变量在第$j,k$个水平上,被试$i$的分数\\ 
        $\mu$          & - &    总体平均值或真值\\
        $\alpha _j$    & - &    自变量水平$j$的处理效应\\
        $\beta _k$     & - &    无关变量$B$水平$k$的变异\\
        $\gamma _l$    & - &    无关变量$C$水平$l$的变异\\
        $\varepsilon _{pooled}$ &-& 合并误差
    \end{tabular}
}

\end{definition}

模型中的$\varepsilon _{pooled}$,pooled指的是“集合的,集中的”,这是两项误差的合并,这点我在后面会介绍.

\subsubsection{单因素拉丁方实验设计适合检验的假说}
(1)处理水平的总体平均数相等,即:
\[ H_0 : \mu _{1..} = \mu _{2..} = \cdots = \mu _{p..}\]

或因素$A$效应等于0,即:
\[ H_0 : \alpha _j = 0 \]

(2)无关变量(横行)的总体平均数相等,即
\[ H_0 : \mu _{.1.} = \mu _{.2.} = \cdots = \mu _{.p.} \]

或无关变量$B$的效应等于0,即
\[ \beta _k = 0 \]

(3)无关变量(纵行)的总体平均数相等,即
\[ H_0 : \mu _{..1} = \mu _{..2} = \cdots = \mu _{..p} \]

或无关变量$C$的效应等于0,即
\[ \gamma _l = 0 \]


 \subsection{实验设计与计算举例}
 \subsubsection{研究的问题与实验设计}
我在做4种文章的生字密度对学生阅读理解影响时,从4个班级随机选取32名学生,每个班8人,实验在周三、四、五、六下午分4次进行.在这个研究中,自变量——生字密度有$a_1,a_2,a_3,a_4$4个水平,考虑到来自不同班级的学生可能在阅读理解方面存在差异,从而影响实验结果,但班级的差异又不是研究者感兴趣的,可以把班级做为一个带有$b_1,b_2,b_3,b_4$4个水平的无关变量;另外,实验时间的不同也可能影响学生的情绪,从而影响实验结果,因此可将实验时间做为第2个无关变量,它也有4个水平:$c_1,c_2,c_3,c_4$ .实验实施前,我需要首先建构一个$4 \times 4$的拉丁方标准块,将每个班级的8名学生随机分配在$c_1,c_2,c_3,c_4$的拉丁方格中,每个方格中的两个学生接受完全相同的实验条件.然后,将拉丁方格标准块随机化,并按随机块的方案实施实验.举例如下:


 
 \subsubsection{实验数据及其计算}

\textbf{1.计算表}

%---ABCS表--------------------------------------------------------------
\begin{margintable}
  \centering
  \caption{拉丁方设计$ABCS$表}
    \begin{tabular}{ccccc}
    \toprule
          & $c_1$ & $c_2$ & $c_3$ & $c_4$ \\
    \midrule
          & $a_1$ & $a_2$ & $a_3$ & $a_4$ \\
    \multirow{2}[0]{*}{$b_1$} & \cellcolor[rgb]{ .949,  .949,  .949}3 & \cellcolor[rgb]{ .949,  .949,  .949}2 & \cellcolor[rgb]{ .949,  .949,  .949}6 & \cellcolor[rgb]{ .949,  .949,  .949}9 \\
          & \cellcolor[rgb]{ .949,  .949,  .949}4 & \cellcolor[rgb]{ .949,  .949,  .949}3 & \cellcolor[rgb]{ .949,  .949,  .949}5 & \cellcolor[rgb]{ .949,  .949,  .949}8 \\
          & $a_2$ & $a_3$ & $a_4$ & $a_1$ \\
    \multirow{2}[0]{*}{$b_2$} & \cellcolor[rgb]{ .949,  .949,  .949}8 & \cellcolor[rgb]{ .949,  .949,  .949}3 & \cellcolor[rgb]{ .949,  .949,  .949}4 & \cellcolor[rgb]{ .949,  .949,  .949}7 \\
          & \cellcolor[rgb]{ .949,  .949,  .949}7 & \cellcolor[rgb]{ .949,  .949,  .949}2 & \cellcolor[rgb]{ .949,  .949,  .949}3 & \cellcolor[rgb]{ .949,  .949,  .949}6 \\
          & $a_3$ & $a_4$ & $a_1$ & $a_2$ \\
    \multirow{2}[0]{*}{$b_3$} & \cellcolor[rgb]{ .949,  .949,  .949}8 & \cellcolor[rgb]{ .949,  .949,  .949}12 & \cellcolor[rgb]{ .949,  .949,  .949}5 & \cellcolor[rgb]{ .949,  .949,  .949}6 \\
          & \cellcolor[rgb]{ .949,  .949,  .949}9 & \cellcolor[rgb]{ .949,  .949,  .949}13 & \cellcolor[rgb]{ .949,  .949,  .949}6 & \cellcolor[rgb]{ .949,  .949,  .949}4 \\
          & $a_4$ & $a_1$ & $a_2$ & $a_3$ \\
    \multirow{2}[0]{*}{$b_4$} & \cellcolor[rgb]{ .949,  .949,  .949}5 & \cellcolor[rgb]{ .949,  .949,  .949}8 & \cellcolor[rgb]{ .949,  .949,  .949}12 & \cellcolor[rgb]{ .949,  .949,  .949}7 \\
          & \cellcolor[rgb]{ .949,  .949,  .949}4 & \cellcolor[rgb]{ .949,  .949,  .949}7 & \cellcolor[rgb]{ .949,  .949,  .949}11 & \cellcolor[rgb]{ .949,  .949,  .949}5 \\
        \bottomrule
    \end{tabular}
  \labtab{latin_square_ABCS_tab}
\end{margintable}

%-----------------------------------------------------------------

%---ABC表
\begin{margintable}
  \centering
  \caption{拉丁方设计$ABC$表}

    \begin{tabular}{ccccc|c}
    \toprule
          & $c_1$ & $c_2$ & $c_3$ & $c_4$ & $\sum$ \\
    \midrule
          & $n=2$ &       &       &       &  \\
          & $a_1$ & $a_2$ & $a_3$ & $a_4$ &  \\
    $b_1$ & \cellcolor[rgb]{ .886,  .937,  .855}7 & \cellcolor[rgb]{ .886,  .937,  .855}5 & \cellcolor[rgb]{ .886,  .937,  .855}11 & \cellcolor[rgb]{ .886,  .937,  .855}17 & \cellcolor[rgb]{ 1,  .949,  .8}40 \\
          & $a_2$ & $a_3$ & $a_4$ & $a_1$ &  \\
    $b_2$ & \cellcolor[rgb]{ .886,  .937,  .855}15 & \cellcolor[rgb]{ .886,  .937,  .855}5 & \cellcolor[rgb]{ .886,  .937,  .855}7 & \cellcolor[rgb]{ .886,  .937,  .855}13 & \cellcolor[rgb]{ 1,  .949,  .8}40 \\
          & $a_3$ & $a_4$ & $a_1$ & $a_2$ &  \\
    $b_3$ & \cellcolor[rgb]{ .886,  .937,  .855}17 & \cellcolor[rgb]{ .886,  .937,  .855}25 & \cellcolor[rgb]{ .886,  .937,  .855}11 & \cellcolor[rgb]{ .886,  .937,  .855}10 & \cellcolor[rgb]{ 1,  .949,  .8}63 \\
          & $a_4$ & $a_1$ & $a_2$ & $a_3$ &  \\
    $b_4$ & \cellcolor[rgb]{ .886,  .937,  .855}9 & \cellcolor[rgb]{ .886,  .937,  .855}15 & \cellcolor[rgb]{ .886,  .937,  .855}23 & \cellcolor[rgb]{ .886,  .937,  .855}12 & \cellcolor[rgb]{ 1,  .949,  .8}59 \\
    $\sum$ & \cellcolor[rgb]{ .839,  .863,  .894}48 & \cellcolor[rgb]{ .839,  .863,  .894}50 & \cellcolor[rgb]{ .839,  .863,  .894}52 & \cellcolor[rgb]{ .839,  .863,  .894}52 & \cellcolor[rgb]{ .867,  .922,  .969}202 \\
    \bottomrule
    \end{tabular}

  \labtab{latin_square_ABC_tab}
\end{margintable}


%---------A表
\begin{margintable}
  \centering
  \caption{拉丁方设计$A$表}
    \begin{tabular}{cccc}
    \toprule
    \multicolumn{1}{c}{$a_1$} & \multicolumn{1}{c}{$a_2$} & \multicolumn{1}{c}{$a_3$} & \multicolumn{1}{c}{$a_4$} \\
    \midrule
    \multicolumn{1}{l}{$np=8$} &       &       &  \\
    \rowcolor[rgb]{ .957,  .69,  .518} 35    & 31    & 56    & 80 \\
    \bottomrule
    \end{tabular}
  \labtab{latin_square_A_tab}
\end{margintable}

\textbf{2.各种基本量的计算}

\begin{align*}
    \sum\limits_{i=1}^{n}\sum\limits_{k=l}^{p}\sum\limits_{l=1}^{p}Y_{ijkl}&=3+4+\cdots=202.000\\
    \frac{\sum\limits_{l=1}^p{\sum\limits_{k=1}^p{\sum\limits_{i=1}^n{Y_{ijkl}}}}}{np^2}=\left[ Y \right] &=\frac{\left( 202 \right) ^2}{\left( 2 \right) \left( 4 \right) ^2}\\
    \sum\limits_{l=1}^p{\sum\limits_{k=1}^p{\sum\limits_{i=1}^n{\left( Y_{ijkl}^{2} \right)}}}&=\left[ ABCS \right] =\left( 3 \right) ^2+\left( 4 \right) ^2+\cdots =1544.000\\
    \sum\limits_{l=1}^p{\sum\limits_{k=1}^p{\frac{\left( \sum\limits_{i=1}^n{Y_{ijkl}} \right) ^2}{n}}}&=\left[ ABC \right] =\frac{\left( 7 \right) ^2}{2}+\frac{\left( 5 \right) ^2}{2}+\cdots =1533.000\\
     &=\left[ A \right] =\frac{\left( 35 \right) ^2}{8}+\frac{\left( 31 \right) ^2}{8}+\cdots =1465.250\\
    \sum\limits_{k=1}^p{\frac{\left( \sum\limits_{l=1}^p{\sum\limits_{i=1}^n{Y_{ijkl}}} \right) ^2}{np}}&=\left[ B \right] =\frac{\left( 40 \right) ^2}{8}+\frac{\left( 40 \right) ^2}{8}+\cdots =1331.250\\
    \sum\limits_{l=1}^p{\frac{\left( \sum\limits_{k=1}^p{\sum\limits_{i=1}^n{Y_{ijkl}}} \right) ^2}{np}}&=\left[ C \right] =\frac{\left( 48 \right) ^2}{8}+\frac{\left( 50 \right) ^2}{8}+\cdots =1276.500\\
\end{align*}


\textbf{3.平方和分解与计算}
\begin{definition}[单因素拉丁方设计平方和分解]
\labdef{one_way_latin_variance}
    \begin{alignat*}{4}
       & SS_{\text{总变异}} & &= SS_{\text{处理间}} & & +SS_{\text{处理内}}\\
       &                    & &= SSA               & & +\left( SSB + SSC + SS_{\text{单元内}} + SS_{\text{残差}}  \right)
    \end{alignat*}
\end{definition}

\begin{alignat*}{3}
    & SS_{\text{总变异}} && =[ABCS]-[Y] && =268.875\\
    & SSA               &&  =[A]   - [Y] && =190.125\\
    &SSB               && =[B] -[Y]&&=56.125\\
    & SSC              && =[C]-[Y] &&= 1.375\\
    & SS_{\text{残差}} && = {[ABC]-[Y]}-SSA-SSB-SSC && =10.250\\
    & SS_{\text{单元内}} && = SS_{\text{总变异}}-SSA-SSB-SSC-SS_{\text{残差}} && = 11.000
\end{alignat*}


\textbf{4.方差分析表及对结果的解释}

\begin{table}[h]
	\centering
	\caption{单因素拉丁方实验的方差分析表}
	\labtab{one_way_latin_ANOVA_Tab}
	{
		\begin{tabular}{lrrcrrr}
			\toprule
			\multicolumn{1}{c}{变异源} & \multicolumn{1}{c}{$SS$} & \multicolumn{1}{c}{$df$} & \multicolumn{1}{c}{$MS$} & \multicolumn{1}{c}{$F$} & \multicolumn{1}{c}{$p$} \\
			\midrule
			A(生字密度) & 190.125 & $p-1=3$ & 63.375 & 25.170 & $<$.001  \\
			B(班级) & 56.125 & $p-1=3$ & 18.708 & 27.19 & $<$.001   \\
			C(实验时间) & 1.375 & $p-1=3$ & 0.458 & 0.67 &  0.58  \\
			残差 & 52.875 & $(p-1)(p-2)=6$ & 1.708 & 2.48 & 0.18\\
			单元内误差 & 11.000 & $p^2(n-1)=16$ & 0.688\\
			\midrule
			总计 & 268.875 & $np^2-1=31$ & & &\\
			\bottomrule
			% \addlinespace[1ex]
			% \multicolumn{6}{p{0.5\linewidth}}{\textit{Note.} Type III Sum of Squares} \\
		\end{tabular}
	}
\end{table}

\textbf{5.平方和与自由度的分解}

\subsection{一些解释}

\subsection{评价}
交互作用难以保证,毕竟三个因素
无关变量水平数要和自变量水平数相等,也难经保证=-=
给我们很好的思想,但是不是特别常用