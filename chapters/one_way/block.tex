\section{单因素随机区组实验设计}

\subsection{基本特点}
行为科学研究中,被试的个体差异是误差变异的重要来源.它常常会混淆实验处理效应,因此是无关变异.

方差分析用随机误差代表一些随机发生的因素带来的影响,如果某个效应带来的变异显著大于随因素带来的变异,我们认为这个效应不同于随机因素,是个在统计上有显著意义的效应.然而,在完全随机实验设计中,这个误差的衡量方式是所有被试间的个体差异,但是被试间的个体误差不能完全代表随机误差,因变量对其中某些因素是敏感的.在这些个体差异带来的无关变异中,随机区组设计找出影响因变量最大的,分离这个无关变量带来的变异,使误差效应变小,使这部分无关变异既不出现在处理效应中混淆处理作用的大小,也不出现在误差中降低检验力.这样在$F$检验中:
\[F=\frac{MSA}{MS_{\text{组内}}}\]

将$MSE$中由无关变量$B$带来的无关变异减去,得到
\[MSE=MS_{组内}-MSB\]

使误差效应更小,$F$更容易显著.

单因素随机区组设计适用于这样的情境:研究中有一个自变量,自变量有两个或多个水平$\left( P \geq 2 \right)$,研究中还有一个无关变量,也有两个或多个水平$\left( n \geq 2 \right)$,并且自变量的水平与无关变量的水平之间没有交互作用.

当无关变量是\textbf{被试变量}时, 一般首先将被试在这个无关变量上进行匹配,然后将他们随机分配给不同的实验处理.这样,区组内的被试在此无关变量上更加同质,他们接爱不同的处理水平时,可看作不受无关变量的影响,主要受处理的影响而区组之间的变异反映了无关变量的影响,我们可以利用方差分析技术区分出这一部分变异,经减少误差变异,获得对应效应的更精确估价.

另外环境因素也是潜在可考虑的区组变量,例职,每天的时间、每年的季节、地点、仪器等方面的因素也可以进行区组,经减少误差变异,\textbf{时间}是一个特别有效的区组变量,因为它常常还会带来一些附加自变量,如身体的生理周期、疲劳等.



\begin{margintable}
	\centering
	\caption{单因素随机区组实验设计中被试的分配}
	\labtab{one_way_ANOVA_block_subject}
	{
			\begin{tabular}{ccccc}
			\toprule
			 & $a_1$ & $a_2$ & $a_3$ & $a_4$ \\
			\midrule
			区组1 & $S_1$ & $S_2$ & $S_3$ & $S_4$ \\
			区组2 & $S_5$ & $S_6$ & $S_7$ & $S_8$ \\
			区组3 & $S_9$ & $S_{10}$ & $S_{11}$ & $S_{12}$ \\
			区组4 & $S_{13}$ & $S_{14}$ & $S_{15}$ & $S_{16}$ \\
			\bottomrule
		\end{tabular}	
	}
\end{margintable}

单因素随机区组实验设计中被试与处理的分配如\vreftab{one_way_ANOVA_block_subject},图中可以看出实验中有一个自变量,该自变量有4个水平.实验中还有一个无关变量,将16个被试在无关变量上进行匹配,分为4个区组,每个区组内4个同质被试,随机分配每个被试接受一个处理水平.

单因素随机区组的下标字母与单因素完全随机差不多,不同的是,完全随机设计中,每个处理内被试有多个,故$i$表示的是处理内第$i$个被试,每个处理都有$n$个被试;而在随机区组设计中,不存在这种处理内的被试,我们看到\vreftab{one_way_ANOVA_block_subject}中$a_1$下的4个被试又属于不同的区组.因此,现在的处理内的被试换成了区组,故一共有$n$个区组,$i$表示第$i$个区组.

\subsubsection{单因素随机区组实验设计模型}

\begin{definition}[单因素随机区组设计模型]
\labdef{one_way_ANOVA_block_model}
\begin{align*}
    Y_{ij} = \mu + \alpha _j +  & \pi _i + \varepsilon _{i\left(j\right)}\\
                                & \left( j=1,2,\cdots , p;i = 1,2,\cdots,n \right)
\end{align*}
其中

{
    \renewcommand\arraystretch{1.25}
    \begin{tabular}{lcl}
        $Y_{ij}$ & - & 被试在区组$i$和处理水平$j$上的分数 \\
        $\mu$ & - & 总体平均数 \\
        $\alpha _j$ & - & 水平$j$的处理效应 \\
        $\pi _i ^2$ & - & 区组效应,且$\pi _i ^2\sim N\left(0,\sigma _\pi ^2\right)$\\
        $\varepsilon_{i \left( j \right)}$ & - & 误差效应 \\
    \end{tabular}
}
\end{definition}


%
%
%
%
\subsubsection{单因素随机区组实验设计检验的假说}



1.处理水平的总体平均数相等
\[ \mu _{.1} = \mu _{.2} = \cdots = \mu _{.p} \]

或处理效应等于0,即
\[ \alpha _j = 0 \]

2.区组的总体平均数相等
\[ \mu _{1.} = \mu _{2.} = \cdots = \mu _{n.} \]

或区组效应等于0,即
\[ H_0 : \pi _i ^2 = 0\]

因随机区组设计中假设区组变量和自变量间没有交互作用,所以在假设中就没有关于交互作用假设.

\subsection{实验设计和计算举例}
\subsubsection{研究问题}
我们仍利用第一节中文章的生字密度对阅读理解影响的研究做例子,由于考虑到学生的智力可能对阅读理解测验分数产生影响,但它又不是该实验中感兴趣的因素,我决定把学生的智力作为一个无关变量,通过实验设计将它的效应分离出去,以更好地探讨生字密度对阅读理解的影响.我选用了单因素随机区组实验设计.这时,我的研究假说、实验的自变量、因变量都是不变的,只是增加了一个无关变量.在实验实施前,我首选要给32个学生做智力测验,并按智力将学生分为8个区组,然后随机分配每个区组内的4个同质被试分别阅读一种生字密度的文章.

\subsubsection{数据计算}

\textbf{1.计算表}
%----------------------------单因素随机区组设计计算表
\begin{margintable}
	\centering
	\caption{单因素随机区组实验的$AS$表}
	\labtab{one_way_ANOVA_block_AS_TAB}
	{
		\begin{tabular}{ccccc|c}
			\toprule
    			 & $a_1$ &  $a_2$ &  $a_3$ &  $a_4$ & $\sum$ \\
    			     \midrule
                        区组1 & \cellcolor[rgb]{ .949,  .949,  .949}3 & \cellcolor[rgb]{ .949,  .949,  .949}4 & \cellcolor[rgb]{ .949,  .949,  .949}8 & \cellcolor[rgb]{ .949,  .949,  .949}9 & \cellcolor[rgb]{ .988,  .894,  .839}24 \\
                        区组2 & \cellcolor[rgb]{ .949,  .949,  .949}6 & \cellcolor[rgb]{ .949,  .949,  .949}6 & \cellcolor[rgb]{ .949,  .949,  .949}9 & \cellcolor[rgb]{ .949,  .949,  .949}8 & \cellcolor[rgb]{ .988,  .894,  .839}29 \\
                        区组3 & \cellcolor[rgb]{ .949,  .949,  .949}4 & \cellcolor[rgb]{ .949,  .949,  .949}4 & \cellcolor[rgb]{ .949,  .949,  .949}8 & \cellcolor[rgb]{ .949,  .949,  .949}8 & \cellcolor[rgb]{ .988,  .894,  .839}24 \\
                        区组4 & \cellcolor[rgb]{ .949,  .949,  .949}3 & \cellcolor[rgb]{ .949,  .949,  .949}2 & \cellcolor[rgb]{ .949,  .949,  .949}7 & \cellcolor[rgb]{ .949,  .949,  .949}7 & \cellcolor[rgb]{ .988,  .894,  .839}19 \\
                        区组5 & \cellcolor[rgb]{ .949,  .949,  .949}5 & \cellcolor[rgb]{ .949,  .949,  .949}4 & \cellcolor[rgb]{ .949,  .949,  .949}5 & \cellcolor[rgb]{ .949,  .949,  .949}12 & \cellcolor[rgb]{ .988,  .894,  .839}26 \\
                        区组6 & \cellcolor[rgb]{ .949,  .949,  .949}7 & \cellcolor[rgb]{ .949,  .949,  .949}5 & \cellcolor[rgb]{ .949,  .949,  .949}6 & \cellcolor[rgb]{ .949,  .949,  .949}13 & \cellcolor[rgb]{ .988,  .894,  .839}31 \\
                        区组7 & \cellcolor[rgb]{ .949,  .949,  .949}5 & \cellcolor[rgb]{ .949,  .949,  .949}3 & \cellcolor[rgb]{ .949,  .949,  .949}7 & \cellcolor[rgb]{ .949,  .949,  .949}12 & \cellcolor[rgb]{ .988,  .894,  .839}27 \\
                        区组8 & \cellcolor[rgb]{ .949,  .949,  .949}2 & \cellcolor[rgb]{ .949,  .949,  .949}3 & \cellcolor[rgb]{ .949,  .949,  .949}6 & \cellcolor[rgb]{ .949,  .949,  .949}11 & \cellcolor[rgb]{ .988,  .894,  .839}22 \\
                        \midrule
                              $\sum$ & \cellcolor[rgb]{ .886,  .937,  .855}35 & \cellcolor[rgb]{ .886,  .937,  .855}31 & \cellcolor[rgb]{ .886,  .937,  .855}56 & \cellcolor[rgb]{ .886,  .937,  .855}80 & \cellcolor[rgb]{ .867,  .922,  .969}202 \\

			\bottomrule
		\end{tabular}
	}
\end{margintable}
%----------------------------------------------------------------------------
\textbf{2.各种基本量的计算}
    \begin{align*}
        %----------------------------------------------------------
        %[Y]
            \frac{
                \left(
        	\sum\limits_{i=1}^{n} \sum\limits_{j=1}^{p}Y_{ij}
                \right)^2}
            {np}=& \colorbox[rgb]{ .867,  .922,  .969}{$[Y]$} = 
            \frac{
                \left(
        	    202
                \right)^2}
                {
                    \left(
        	           8
                    \right)^2
                    \left(
                        4
                    \right)^2
                }&=1275.125\\
        %----------------------------------------------------------
        %[AS]
            \sum\limits_{i=1}^{n} \sum\limits_{j=1}^{p}Y_{ij}^2=
            & \colorbox[rgb]{ .949,  .949,  .949}{$[AS]$} = 
            \left(
            	3
            \right)^2 +
            \left(
            	6
            \right)^2    +\cdots &=1544.000\\
        %----------------------------------------------------------
        %[A]    
            \sum\limits_{i=1}^{n}
            \frac
                {\left(
	            \sum\limits_{j=1}^{p}Y_{ij}
                \right)^2}
                {n}=
            &\colorbox[rgb]{ .886,  .937,  .855}{$[A]$}=
            \frac{\left(35\right)^2}{8}+\frac{\left(31\right)^2}{8}+\cdots&=1465.250\\
        %---------------------------------------------------------
        %[S]
            \sum\limits_{i=1}^{n}
            \frac
            {
                \sum\limits_{j=1}^{p}\left(Y_{ij} \right)^2
            }
            {
                p            
            }            
            =&\colorbox[rgb]{ .988,  .894,  .839}{$[S]$}
            =\frac{\left( 24 \right)^2}{4} + \frac{\left( 29 \right)^2}{4} + \cdots &= 1301.000
    \end{align*}
%---------------------------------------------
\textbf{3.平方和的分解}

\begin{definition}[单因素随机区组设计平方和的分解]
\labdef{one_way_ANOVA_block_variance_breakthrough}
\begin{alignat*}{2}
   & SS_{\text{总变异}} &&= SS_{\text{处理间}} + SS_{\text{处理内}}\\
   &                    &&= SSA + (SS_{\text{区组}} + SS_{\text{残差}})
\end{alignat*}
\end{definition}
\begin{alignat*}{3}
    &    SS_{\text{总变异}} &&     =[AS]-[Y]                                 && =268.875\\
    &    SSA                &&    =[A]-[Y]                                  && =190.125\\
    &    SS_{\text{区组}}   &&    =[S]-[Y]                                   && =25.750\\
    &    SS_{\text{残差}}   &&    =SS_{\text{总变异}}-SSA-SS_{\text{区组}}    &&=52.875
\end{alignat*}
%----------------------
\textbf{4.方差分析表及结果解释}
\begin{table}[h]
	\centering
	\caption{单因素随机区组实验的方差分析表}
	\labtab{one_way_block_ANOVA_Tab}
	{
		\begin{tabular}{lrrcrrr}
			\toprule
			\multicolumn{1}{c}{变异源} & \multicolumn{1}{c}{$SS$} & \multicolumn{1}{c}{$df$} & \multicolumn{1}{c}{$MS$} & \multicolumn{1}{c}{$F$} & \multicolumn{1}{c}{$p$} \\
			\midrule
			$A$(生字密度) & 190.125 & $p-1=3$ & 63.375 & 25.170 & $<$ .001  \\
			区组(智力) & 25.850 & $n-1=7$ & 3.696 & 1.47 & 0.23   \\
			残差 & 52.875 & $(n-1)(p-1)=21$ & 2.518\\
			\midrule
			总计 & 268.875 & $np-1=31$ & & &\\
			\bottomrule
			% \addlinespace[1ex]
			% \multicolumn{6}{p{0.5\linewidth}}{\textit{Note.} Type III Sum of Squares} \\
		\end{tabular}
	}
\end{table}

方差分析可以看出,实验中的自变量——生字密度的效应是统计显著的$\left( F(3,21)=25.170, p < .001 \right)$,说明学生对生字密度不同的文章的阅读理解有显著差别.实验中无关变量——智力的效应是统计上不显著的$\left( F \left( 7,21 \right) = 1.47, p > .05 \right)$,表明本实验中智力不同的学生的阅读理解 没有明显差异.方差分析表中还可以看出,生字密度和智力的$F$检验都使用了同一个误差项$MSE=2.518$.

\textbf{5.平方和与自由度分解图}


\subsection{一些解释}
\textbf{1.各种平方和的意义}

\begin{description}
\item[$SS_{\text{总变异}}$]——在随机区组实验中,总平方和应首先分解为处理间平方和处理内平方和应首先分解 为处理间平方和处理内平方和.
\item[$SS_{\text{处理间}}$]——处理间平方和指所有由实验处理引起的变异,在单因素设计中指$A$因素的处理效应$SSA$
\item[$SS_{\text{处理内}}$]——在随机区组实验中,处理内平方和可进一步分解为两部分:区组平方和和误差平方和.
\item[$SS_{\text{区组}}$]——区组效应,在该实验中指总变异中由被试的智力引起的变异.
\item[$SS_{\text{残差}}$]——残差指总变异中不能被实验处理和区组效应解释的变异.在随机区组实验设计中,接受同实验条件的同质被试只有一个,因此,不能计算单元内误差,而残差作为误差变异的估计.残差的计算是从总变异中减去处理效应和区组变异.
\end{description}

\textbf{2.做区组的方法}

除了感兴趣的自变量外,还有很多无关就量都影响因变量.如果发现一个变量特别影响因变量,但对其效应不感兴趣,可以考虑将其作一个区组.本例中区组变量是智力,我们知道一个孩子的智力情况会影响他的阅读理解,如果一个组里都是智力低的孩子,一个组里都是智力高的孩子.若智力高的组分配的是低生字密度,生字密度和智力都会促进阅读理解,处理效应被加大,当本身处理效应不显著时,会面临错误判定处理效应显著的失误;若智力高的组分配的是高生字密度,自变量效应方向和无关变量方向相反,它们抵消后会使处理效应被掩盖,造成处理效应不显著.

\textbf{3.随机区组提高实验敏感性的体现}

前面我提到从$F$值的大小思考实验的敏感性,单因素完全随机的$F$值:
\[ F=\frac{MSA}{MS_{\text{组内}}} \]

单因素随机区组设计将$SS_{组内}$进一步分解为$SS_{\text{区组}}$和$SS_{\text{残差}}$,既然区组变量带来的变异不是随机误差,因此将其从组内误差中分离出来,这样误差项就减小了,从而单因素随机区组的$F$值就是:
\[ F = \frac{MSA}{MS_{\text{残差}}} \]

而$MS_{\text{残差}} < MS_{\text{组内}}$,因而提升了实验的敏感性.从数据上,我们可以看到,在\vreftab{one_way_ANOVA_Tab}中,组内误差是78.750,在同样的数据用了随机区组的算法后,\vreftab{one_way_block_ANOVA_Tab}中可以看出,将组内误差分解成了区组带来的无关变异25.850,和剩余的残差52.875.完全随机的$F$是22.533,在随机区组中$F$是25.170,这看上去没有提升多少,原因是这组数据中区组变量的效应是不显著的,可以看到残差项还是很大,说明这个随机区组变量并不是特别有效.

那么什么是有效的随机区组设计呢?
在\vreftab{one_way_ANOVA_AS_TAB}中右边红色的数据是对把自变量处理水平$a_1,a_2,a_3,a_4$合并起来的数据,这样这个数据只和集中趋势和区组无关变量带来的变异相关,我们把它按小到大排个序:
\[ 19,22,24,24,26,27,29,21 \]

这和绿色的数据:
\[ 31,35,56,80 \]

相比,变异就小了很多.一个好的区组设计应该让红色的那部分数据呈现一个明显的梯度,让它们之间相差得大一点,这样就可以带走更多变异.这点不是实验操作上可以保证的,主要是要看这个区组变量是不是一个有效的区组变量,是否对因变量有较大影响.

\textbf{4.残差的实质}

从方差分析中的自由度(见\vreftab{one_way_block_ANOVA_Tab})知,残差的自由度是$(p-1)(n-1)$,这个形式表明这个残差本质是个交互作用.现在还不是时候从根本上揭示残差的实质.

随机区组中,用残差来估价随机误差,今后我们将看到,所有的误差项只有两种形式,一种是单元内误差(当有同质被试接受相同的处理时产生),一种是残差(实验是交互作用).在随机区组设计中,由于没有接受相同处理的同质被试,故不存在单元内误差.

\subsection{评价}
随机区组的\textbf{优点}是,在许多研究情境中,它比完全随机实验设计更加有效.这是由于它使研究者从总变异中分离出了一个无关变量的效应,从而减小了实验误差,可获得对处理效应的更加精确的估价.随机区组实验设计可使用于含任何处理水平数的实验中,并且区组的数量也不受限制,因而有较好的灵活性.

随机区组实验设计的\textbf{缺点}是,如实验中含有多处理水平,可能给形成同质区组、寻找同质被试带来困难.另外,使用随机区组设计比使用完全随机设计有更多的限定,例如使用随机区组实验设计的前提假设是,实验中的自变量与无关变量之间没有交互作用.如果交互作用是存在的,使用随机区组实验设计是不合适的.这在一定程度上限制了随机区组实验设计的应用.但由于无法对该残差的显著性进行检验,故对于这个误差的估计是否合理就没有检验的方法.

\subsection{经典随机区组设计的改进}
\subsubsection{基本思想}
经典随机区组设计中,误差用残差衡量,它的实质是区组变量与自变量间的交互作用,问题是我们无法保证它们间一定不存在交互作用,我们甚至没有检验这个交互作用显著性的手段.

单元内误差是一定是一个可靠的误差估计,不过在经典随机区组设计中没有接受相同处理的同质被试,那对其进行改进即可,让区组内人数扩大,使接受同一个处理和在同一个区组内的被试有多个,这样可以算出单元内误差,检验残差是否显著.

\subsubsection{评价}
我们力求区组内被试同质,但是实际上很难做到.本例来说,区组变量是智力,这是一个连续变量.增加区组内被试时,有很大可有使区组内的变异变大,很有可能让区组间的变小.

所以随机区组设计不会用这种方式.