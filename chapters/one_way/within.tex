\section{单因素被试内实验设计}

\subsection{基本特点}
在单因素完全随机实验中,组内变异实际上由两部分组成的:实验中测量误差引起的变异和未控制的无关变量带来的变异,其中主要是被试个体差异带来的变异.减少误差变异的一个方法是控制个体差异引起的无关变异,达互这个目标的途径之一是使用随机区组设计,带走一个无关变量引起的无关变异;另一个方法是使用拉丁方设计,带走两个无关变量引起的无关变量;
因个体差异是无关变量引起的无关变异的主要来源,故一个更有效的方法是重要测量实验设计,在单因素实验设计中,重复测量设计也是被试内设计,这种设计的基本思路是分离出所有被试间个体差异.

前面介绍的完全随机设计、随机区组设计、拉丁方设计中,每一个被试都只接受一个处理,这种设计被称为被试间实验设计或非重复测量设计.完全随机设计中,误差用所有被试间的变异估价,随机区组设计和拉丁方设计进一步将其中的一个或两个无关变量带来的无关变异带走,可不论怎么样,总还是有被试个体差异混杂在误差变异中.

重复测量实验设计的基本方法是:实验中每个被试接受所有的处理水平.这种实验设计的目的是利用被试自己做控制,使被度的各方面特点在所有的处理中保持恒定,经最大限度地控制由被试的个体差异带来的变异.

使用重复测量设计的前提是研究者必须事先假设,当若干处理水一连续实施给同一被试时,被试接受前面的处理对接受后面的处理没有长期影响.重复测量设计在有些情况下是不合适的,当处理的实施对被试有长期影响时,如学习、记忆效应,不能使用重复测量设计.例如,在一个教学研究中,要比较两种教学方法对学生成绩的影响.我们不可能使用同一班学生先后接受两种教学方法,然后比较它们对学生学习成绩的影响,因为前一种教法的教学不可避免地对学生接受后一种教法的教学产生影响.在心理与教育研究中,话多实验处理会对被试产生学习、记忆效应,因此用重复测量设计要特别谨慎.

另外顺序效应也是重复测量设计中应特别注意的问题.被试连续接受处理时,练习、疲劳等效应是难免的,因此重复测量中需要考虑平衡顺序效应的问题.

\subsubsection{单因素被试设计模型}

\begin{definition}[单因素被试设计模型]
\labdef{one_way_repeated_model}
    \begin{align*}
        Y_{ij} = \mu + \alpha _j & + \pi _i + \left( \alpha \pi \right)_{ij} +  \varepsilon _{ij}\\
                                   &\left( j=1, 2, \cdots ,p; i=1,2, \cdots , n \right)
    \end{align*}
    
    其中
    
    {
    \renewcommand\arraystretch{1.25}
    \begin{tabular}{ccl}
        $Y_{ij}$             & - &    两无关变量在第$j,k$个水平上,被试$i$的分数\\ 
        $\mu$                & - &    总体平均值或真值\\
        $\alpha _j$          & - &    自变量水平$j$的处理效应\\
        $\pi _i$             & - &    无关变量$C$水平$l$的变异\\
        $\left( \pi  \alpha \right)_{ij}$ &-& 水平$\alpha _j$和被试$i$的交互作用\\
        $\varepsilon _{ij}$  & - &    四舍五入剩余的点误差,不体现在平方和分解中
    \end{tabular}
}  
\end{definition}

\subsubsection{单因素重复测量实验设计检验的假说}

处理水平的总体平均数相等,即:

\[ H_0 : \mu _{.1} = \mu _{.2} = \cdots = \mu _{.p}  \]

或处理效应等于这,即

\[ \alpha _j = 0 \]

\subsection{实验设计与计算举例}
\subsubsection{研究的问题与实验设计}

\begin{margintable}
  \centering
  \caption{单因素被试实验设计平衡顺序效应}
    \begin{tabular}{c|c|llll}
    \toprule
        \multicolumn{1}{r}{} &       & \multicolumn{4}{c}{呈现方式} \\
        \cmidrule{3-6}    \multicolumn{1}{r}{} &       & \multicolumn{1}{c}{1} & \multicolumn{1}{c}{2} & \multicolumn{1}{c}{3} & \multicolumn{1}{c}{4} \\
    \midrule
        \multirow{4}[0]{*}{被试} & $S_1, S_2$ & $a_1$ & $a_2$ & $a_3$ & $a_4$ \\
              & $S_3, S_4$ & $a_2$ & $a_3$ & $a_4$ & $a_1$ \\
              & $S_5, S_6$ & $a_3$ & $a_4$ & $a_1$ & $a_2$ \\
              & $S_7, S_8$ & $a_4$ & $a_1$ & $a_2$ & $a_3$ \\
        \bottomrule
    \end{tabular}
  \labtab{one_way_repeated_balance}
\end{margintable}

 我们继续以4种文章的生字密度对学生阅读理解的影响的研究为例.为了更好地控制被试变量,我仅用8名被试,每个被试阅读4篇生字密度不同的文章,并测他们对各篇文章的阅读理解分数.选择使用重复测量实验设计是由于我假设,当实验安排合适时,被试阅读一篇文章不会对阅读另一篇文章产生影响.但是,在这种实验设计中,疲劳效应和顺序效应是必须考虑的.为了减少疲劳效应,研究者决定将4篇文章在四个下午分4次施测.平衡顺序效应的方式有两种:以随机顺序实施4和种生字密度的文章,或以拉丁方排序实施4种生字密度的文章.后一平衡效应的方法举例如\vreftab{one_way_repeated_balance}.
  
\subsubsection{实验数据及其计算}

\textbf{1.计算表}
%----------------------------单因素被试内设计计算表
\begin{margintable}
	\centering
	\caption{单因素被试内设计的$AS$表}
	\labtab{one_way_repeated_ANOVA_AS_TAB}
	{
            \begin{tabular}{ccccc|c}
	    \toprule
                & \multicolumn{1}{l}{$a_1$} & \multicolumn{1}{l}{$a_2$} & \multicolumn{1}{l}{$a_3$} & \multicolumn{1}{l}{$a_4$} & \multicolumn{1}{l}{$\sum$} \\
            \midrule
                被试1   & \cellcolor[rgb]{ .949,  .949,  .949}4 & \cellcolor[rgb]{ .949,  .949,  .949}3 & \cellcolor[rgb]{ .949,  .949,  .949}5 & \cellcolor[rgb]{ .949,  .949,  .949}7 & \cellcolor[rgb]{ .988,  .894,  .839}19 \\
                被试2   & \cellcolor[rgb]{ .949,  .949,  .949}3 & \cellcolor[rgb]{ .949,  .949,  .949}2 & \cellcolor[rgb]{ .949,  .949,  .949}6 & \cellcolor[rgb]{ .949,  .949,  .949}8 & \cellcolor[rgb]{ .988,  .894,  .839}19 \\
                被试3   & \cellcolor[rgb]{ .949,  .949,  .949}3 & \cellcolor[rgb]{ .949,  .949,  .949}5 & \cellcolor[rgb]{ .949,  .949,  .949}7 & \cellcolor[rgb]{ .949,  .949,  .949}11 & \cellcolor[rgb]{ .988,  .894,  .839}26 \\
                被试4   & \cellcolor[rgb]{ .949,  .949,  .949}2 & \cellcolor[rgb]{ .949,  .949,  .949}4 & \cellcolor[rgb]{ .949,  .949,  .949}6 & \cellcolor[rgb]{ .949,  .949,  .949}9 & \cellcolor[rgb]{ .988,  .894,  .839}21 \\
                被试5   & \cellcolor[rgb]{ .949,  .949,  .949}5 & \cellcolor[rgb]{ .949,  .949,  .949}4 & \cellcolor[rgb]{ .949,  .949,  .949}7 & \cellcolor[rgb]{ .949,  .949,  .949}8 & \cellcolor[rgb]{ .988,  .894,  .839}24 \\
                被试6   & \cellcolor[rgb]{ .949,  .949,  .949}5 & \cellcolor[rgb]{ .949,  .949,  .949}4 & \cellcolor[rgb]{ .949,  .949,  .949}8 & \cellcolor[rgb]{ .949,  .949,  .949}12 & \cellcolor[rgb]{ .988,  .894,  .839}29 \\
                被试7   & \cellcolor[rgb]{ .949,  .949,  .949}6 & \cellcolor[rgb]{ .949,  .949,  .949}3 & \cellcolor[rgb]{ .949,  .949,  .949}8 & \cellcolor[rgb]{ .949,  .949,  .949}12 & \cellcolor[rgb]{ .988,  .894,  .839}29 \\
                被试8   & \cellcolor[rgb]{ .949,  .949,  .949}7 & \cellcolor[rgb]{ .949,  .949,  .949}6 & \cellcolor[rgb]{ .949,  .949,  .949}9 & \cellcolor[rgb]{ .949,  .949,  .949}13 & \cellcolor[rgb]{ .988,  .894,  .839}35 \\
            \midrule
                $\sum$ & \cellcolor[rgb]{ .886,  .937,  .855}35 & \cellcolor[rgb]{ .886,  .937,  .855}31 & \cellcolor[rgb]{ .886,  .937,  .855}56 & \cellcolor[rgb]{ .886,  .937,  .855}80 & \cellcolor[rgb]{ .867,  .922,  .969}202 \\
            \bottomrule
	    \end{tabular}
	}
\end{margintable}
%----------------------------------------------------------------------------
单因素被试内设计中的计算表有一张(\vreftab{one_way_repeated_ANOVA_AS_TAB}),这张表和随机区组设计的表很像,不过为了突显被试内设计可以带走更多的无关变异,我对数据进行了一些调整.\vreftab{one_way_repeated_ANOVA_AS_TAB}
中红色的数据相比\vreftab{one_way_ANOVA_block_AS_TAB}中的红色数据呈现一个更加明显的梯度,也就是区组之间的变异小于被试间的变异,而下方绿色的数据仍然不变.之后在结果中可以看到被试的$F$是更大的.

\textbf{2.各种基本量的计算}
    \begin{align*}
        %----------------------------------------------------------
        %[Y]
            \frac{
                \left(
        	\sum\limits_{i=1}^{n} \sum\limits_{j=1}^{p}Y_{ij}
                \right)^2}
            {np}=& \colorbox[rgb]{ .867,  .922,  .969}{$[Y]$} = 
            \frac{
                \left(
        	    202
                \right)^2}
                {
                    \left(
        	           8
                    \right)^2
                    \left(
                        4
                    \right)^2
                }&=1275.125\\
        %----------------------------------------------------------
        %[AS]
            \sum\limits_{i=1}^{n} \sum\limits_{j=1}^{p}Y_{ij}^2=
            & \colorbox[rgb]{ .949,  .949,  .949}{$[AS]$} = 
            \left(
            	3
            \right)^2 +
            \left(
            	6
            \right)^2    +\cdots &=1544.000\\
        %----------------------------------------------------------
        %[A]    
            \sum\limits_{i=1}^{n}
            \frac
                {\left(
	            \sum\limits_{j=1}^{p}Y_{ij}
                \right)^2}
                {n}=
            &\colorbox[rgb]{ .886,  .937,  .855}{$[A]$}=
            \frac{\left(35\right)^2}{8}+\frac{\left(31\right)^2}{8}+\cdots&=1465.250\\
        %---------------------------------------------------------
        %[S]
            \sum\limits_{i=1}^{n}
            \frac
            {
                \sum\limits_{j=1}^{p}\left(Y_{ij} \right)^2
            }
            {
                p            
            }            
            =&\colorbox[rgb]{ .988,  .894,  .839}{$[S]$}
            =\frac{\left( 19 \right)^2}{4} + \frac{\left( 19 \right)^2}{4} + \cdots &= 1330.500
    \end{align*}
    
\textbf{3.平方和的分解与计算}

\begin{definition}[单因素被试内和方分解]
\labdef{one_way_repeated_variance}
\begin{alignat*}{3}
    & SS_{\text{总变异}}      &&  = SS_{\text{被试间}}  && + SS_{\text{被试内}}\\ 
    &                         && = SS_{\text{被试间}}  && + (SSA + SS_{\text{残差}})
\end{alignat*}
\end{definition}

%不让有间隙嘤嘤嘤
\begin{alignat*}{3}
    &    SS_{\text{总变异}}    & & =[AS]-[Y]                               && =268.875\\
    &    SS_{\text{被试间}}     & & =[S]-[Y]                                &&   =55.375\\
    &    SS_{\text{被试内}}     & & =SS_{\text{总变异}}-SS_{\text{被试间}}  &&   =213.500\\
    &    SSA                  & & =[A]-[Y]                                && =190.125\\
    &    SS_{\text{残差}}     & & =SS_{\text{被试内}}-SSA              &&  =23.375
\end{alignat*}

计算时,被试内设计第一步就拿走了被试间个体差异,完全不参与计算.下面的方差分析表
(\vreftab{one_way_repeated_ANOVA_Tab})中也可以看到,第一排就是被试间变异,我们没有检验其显著性,因为这就是一个无关变异,并且它显著或是不显著都不是我们所感兴趣的.

\textbf{4.方差分析表及对结果的解释}
\begin{table}[h]
	\centering
	\caption{单因素被试内实验的方差分析表}
	\labtab{one_way_repeated_ANOVA_Tab}
	{
		\begin{tabular}{lrrrrrr}
			\toprule
			\multicolumn{1}{c}{变异源} & \multicolumn{1}{c}{$SS$} & \multicolumn{1}{c}{$df$} & \multicolumn{1}{c}{$MS$} & \multicolumn{1}{c}{$F$} & \multicolumn{1}{c}{$p$} \\
			\midrule
			被试 & 55.375 & $n-1=7$ & 7.911    \\
			A(生字密度) & 190.125 & $p-1=3$ & 63.375 & 56.936 & $<$ .001  \\
			残差 & 23.375 & $(n-1)(p-1)=21$ & 1.113\\
			\midrule
			总计 & 268.875 & $np-1=31$ & & &\\
			\bottomrule
			% \addlinespace[1ex]
			% \multicolumn{6}{p{0.5\linewidth}}{\textit{Note.} Type III Sum of Squares} \\
		\end{tabular}
	}
\end{table}

方差分析表中可以看出,实验中的自变量——生字密度的效应是统计显著的($F \left( 3,21 \right) = 25.17, p < .01 $),自变量的$F$检验的误差项是$MSE=1.113$.方差分析表中还有一项“被试间变异”($SS_{\text{被试间}}=55.375$),这部分变异事业有所有被试之间个体差异引起的变异.由于从总变异中分享出了被试间变异,因此与完全随机实验相比,重复测量实验中提高了实验处理的$F$检验的敏感性.

\textbf{5.平方和与自由度分解图解}

\subsection{一些解释}

\begin{description}
\item[$SS_{\text{总变异}}$] —— 在重复测量实验中,总变异应首先分解为被试间平方和和被试内平方和
\item [$SS_{\text{被试间}}$] —— 被试间平方和,即总变异中所有由被试的个体差异引起的变异
\item [$SS_{\text{被试内}}$] —— 被试内平方和包括同一被试在接受不同实验处理时产生的变异(铺好床吖啊内因素的处理效应),以有偶然因素引起的实验误差.在单因素重复测量实验设计中,被试内平方和被分解为两个部分:$A$因素的处理效应和误差变异
\item [$SS_{\text{残差}}$]   —— 重复测量实验中的残差变异与随机区组实验中的残差性质相同,重复测量实验方差分析中残差计算是先从总变异中减去被试间平方和,然后再减去处理效应.由于事先从总变异中分离出了所有的由被试个体差异带来的变异,重复测量实验中的$SS_{\text{残差}}$一般很小

\end{description}

\subsection{评价}

